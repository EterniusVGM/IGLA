\newpage
\section{Лекция 12}
\subsection{Итеративные методы решения систем алгебраических уравнений}
Дана система уравнений $$A\bar x=\bar B~~(1)$$
Перепишем ее в виде $$\bar x+(A-E)\bar x=\bar B$$
или $$\bar x=P\bar x+\bar b$$
Тут $P=E-A,~\bar b=\bar B$.\\
Для $c\neq 0:~cA\bar x=c\bar B \Rightarrow \bar x=(E-cA)+c\bar B$, где $P=E-cA,~\bar b=c\bar B$.\\
Если $C$ --- матрица, тогда $P=E-CA,~\bar b=C\bar B$.\\ \\
\textbf{Метод итераций.}\\
Пусть $\bar x^0$ --- любой вектор (начальное приближение к $\bar x$), тогда итеративная формула для вычисления $\bar x^1, \cdots, \bar x^k$ имеет вид
$$x^{k+1}=Px^k+\bar b$$ 
Если последовательность векторов $lim\{\bar x^i\}=\bar x^{\infty}$ сходится, то $$x^{\infty}=Px^{\infty}+\bar b,~x=x^{\infty},$$
где $x=x^{\infty}$ --- решение системы.\\
\\
\begin{theorem}
    Описанный метод простых итераций сходится, то есть существует $\bar x^{\infty}=\underset{i\to \infty}{lim}\{\bar x^i\}$, которое будет решением системы при любом значении $x^0$ тогда и только тогда, когда спектральный радиус $\rho(P)<1$.
\end{theorem}
\begin{proof}
    Пусть спектральный радиус $\rho(P)<1$, тогда существуют согласованные матричные $||\cdot||$ и векторные $|\cdot|$ нормы, что $||P||<1$.\\
Предположим, что решение системы существует. Если $\bar x$ --- решение системы, то \begin{center}
    $x^{k+1}-x=(Px^k+\bar b)-(Px+\bar b)=P(x^k-x)$\\
    $|x^{k+1}-x|\leqslant||P||~|x^k-x|~~(2)$
\end{center}
Значит $|x^{k+1}-x|\leqslant \cdots \leqslant ||P||^{k+1}|x^0-x| \to 0$. То есть, если решение $\bar x$ существует, то $\overset{lim}{i\to \infty}x^i=x$ --- решение системы.\\
Если же решения нет $|\lambda_1|=\rho(P)\geqslant 1$, то существует $\bar v\neq 0:~Pv=\lambda_1 v$ --- собственный вектор, и для $x^0=v+\bar x$ получим
$$x^{k+1}-x=P(x^k-x)=\cdots=P^{k+1}(x^0-x)=P^{k+1}v=\lambda^{k+1}v \nrightarrow 0$$
\end{proof}
Если $||P||<N$, то $|x^k-x|\leqslant N^k|x^0-x|$.\\
Число верных значащих цифр $-c\cdot log_{10}|x^k-x|=\xi(x^k)$, где $c$ зависит от нормы.\\ \\
\begin{proposal}
    $$-const\cdot log_{10}(\rho(P))\leqslant \xi(x^{k+1})-\xi(x^k)$$
Для нормы $|\cdot|_{\infty}$ $const=1$.
\end{proposal}
\begin{proof}
    \ 
\begin{center}
    $\rho(P) \approx ||P|| \geqslant \cfrac{|x^{k+1}-x}{x^k-x}$\\
    $log_{10}(\rho(P))\geqslant log_{10}|x^{k+1}-x|-log_{10}|x^k-x|$\\
    $-log_{10}(\rho(P))\leqslant\cfrac{1}{c}(\xi(x^{k+1}-\xi(x^k))$
\end{center}
\end{proof}
Надо, чтобы спектральный радиус был маленький.
\begin{statement}
    \ 
\begin{center}
    $|x-x^k|\leqslant \cfrac{|x^{k+1}-x^k|}{1-||P||}$, если $||P||<1$
\end{center}
\end{statement}
\begin{proof}
    $$|x^{k+p}-x^k|\leqslant|x^{k+p}-x^{k+p-1}|+\cdots+|x^{k+1}-x^k|\overset{(2)}{\leqslant}(||P||^p+\cdots+||P||+1)|x^{k+1}-x^k|\leqslant\cfrac{|x^{k+1}-x^k}{1-||P||}$$
$$|x-x^k|=|x^{\infty}-x^k|=\underset{p\to \infty}{lim}|x^{k+p}-x^k|\leqslant\cfrac{|x^{k+1}-x^k|}{1-||P||}$$
\end{proof}
\begin{consequence}
    \ 
\begin{center}
    $|x-x^k|\leqslant \cfrac{||P||^k}{1-||P||}|x^1-x^0|$
\end{center}
\end{consequence}
\begin{consequence}
    \ 
\begin{center}
    Если $\bar x^0=\bar b$, то $|x-x^k|\leqslant \cfrac{||P||^{k+1}}{1-||P||}|\bar b|$
\end{center}
\end{consequence}
\begin{proof}
    $x^1=Px^0+b$. Если $x^0=b$, то $x^1=Pb+b$.\\
$$|x^1-x^0|=|Pb|\leqslant||P||~|b|$$
\end{proof}
\textbf{Пример 1.}\\
Методом итераций решить систему.\\ \\
$
\left\{
\begin{array}{lcl}
20x_1+3x_2+x_3=68\\
2x_1+20x_2+3x_3=41\\
3x_1+x_2+10x_3=60\\
\end{array}
\right.
$
\\
\\
Тогда матрица $C$ будет иметь вид
\[C=\begin{pmatrix}
~\cfrac{1}{20}~ & 0 & 0\\
0 & ~\cfrac{1}{20}~ & 0\\
0 & 0 & ~\cfrac{1}{20}~\\
\end{pmatrix}\]
$$CAx=CB,~x=Px+b=(E-CA)x+CB$$
То есть,  
\[P=E-CA=\begin{pmatrix}
0 & -0.15 & -0.05\\
-0.1 & 0 & -0.15\\
-0.3 & -0.1 & 0\\
\end{pmatrix}\]
\[b=CB=\begin{pmatrix}
3.4\\
2.05\\
6\\
\end{pmatrix}\]
\[x=\begin{pmatrix}
x_1\\
x_2\\
x_3\\
\end{pmatrix} = \begin{pmatrix}
0 & -0.15 & -0.05\\
-0.1 & 0 & -0.15\\
-0.3 & -0.1 & 0\\
\end{pmatrix}x + \begin{pmatrix}
3.4\\
2.05\\
6\\
\end{pmatrix}\]
Посчитаем норму матрицы: $||P||_c=0.9<1,~||P||_1=0.4<1$. Так как ее значения меньше 1, значит процесс итераций будет сходящимся.\\
Возьмем в качестве первого приближения 
\[x^0=\begin{pmatrix}
3.4\\
2.05\\
6\\
\end{pmatrix}\]
Подставим это значение в $x^1=Px^0+b$, получим 
\[x^1 \approx \begin{pmatrix}
3\\
1\\
5\\
\end{pmatrix}\]
Теперь вместо $x^0$ подставляем $x^1$ и получаем $x^2$, и так далее, пока значение $x$ не будет больше изменяться.
\subsection{Метод Зейделя}
Этот метод является модификацией метода итераций. 
$$x=Px+b$$
Представим матрицу $P$ в виде суммы верхней и нижней треугольных матриц.
\[P=P_1+P_2= \begin{pmatrix}
0 & \cdots & 0\\
* & \ddots & \vdots\\
* & * & 0\\
\end{pmatrix} + \begin{pmatrix}
p_{11} & * & *\\
\vdots & \ddots & *\\
0 & \cdots & p_{nn}\\
\end{pmatrix}\]
Тогда итеративная формула вычисления будет иметь вид
$$x^{k+1}=P_1x^{k+1}+P_2x^k+b$$
\textbf{Пример 2.}\\
Сделаем пример 1 с помощью метода Зейделя. В нем мы вычислили
\[P=E-CA=\begin{pmatrix}
0 & -0.15 & -0.05\\
-0.1 & 0 & -0.15\\
-0.3 & -0.1 & 0\\
\end{pmatrix}\]
Представим ее в виде суммы верхней и нижней треугольных матриц.
\[P=P_1+P_2=\begin{pmatrix}
0 & 0 & 0\\
-0.1 & 0 & 0\\
-0.3 & -0.1 & 0\\
\end{pmatrix} + \begin{pmatrix}
0 & -0.15 & -0.05\\
0 & 0 & -0.15\\
0 & 0 & 0\\
\end{pmatrix}\]
Тут также за нулевое приближение возьмем
\[x^0=\begin{pmatrix}
3.4\\
2.05\\
6\\
\end{pmatrix}\]
$$x_1'=0\cdot x_1'-(0.15\cdot 2.05+0.05\cdot 6)+3.4=2.7925$$
$$x_2'=-0.1x_1'-0.15\cdot 6+2.05=0.87075$$
$$x_3'=-0.3x_1'-0.1x_2'-0+6=5.075175$$
\textbf{Приведение к виду, удобному для итераций.}\\
$$Ax=b$$
$$\tau Ax=\tau b$$
$$x=(E-\tau A)x+\tau b$$
Надо найти такое оптимальное $\tau$, что $\underset{\tau}{min}\underset{\lambda \in [\lambda_{min},\lambda_{max}]}{max}|1-\tau \lambda|$.
\begin{statement}
    Если все собственные числа матрицы $\lambda_i>0$, то оптимальное значение $\tau=\cfrac{2}{\lambda_{min}+\lambda_{max}}$.
\end{statement}
\noindent\textbf{Пример 3.}\\
Привести к виду, удобному для итераций.\\
$
\left\{
\begin{array}{lcl}
2x_1+2x_2-2x_3=6\\
2x_1+5x_2-4x_3=1\\
-2x_1-4x_2+5x_3=1\\
\end{array}
\right.
$
\\ \\
Собстевнные значения матрицы $\lambda_1=\lambda_2=1,~\lambda_3=10$. Так как все они положительные, получим: 
$$\tau=\cfrac{2}{10+1}=\cfrac{2}{11}$$
\[E-\tau A=\begin{pmatrix}
1-2\tau & -2\tau & 2\tau \\
-2\tau & 1-5\tau & 4\tau \\
2\tau & 4\tau & 1-5\tau \\
\end{pmatrix}\]
Тогда $max\{|1-2\tau|,~|1-3\tau|,~|1+\tau|\}$.
\subsection{Домашнее задание 12}\begin{enumerate}
    \item Доказать утверждение из лекции. Если все собственные числа матрицы $\lambda_i>0$, то оптимальное значение $\tau=\cfrac{2}{\lambda_{min}+\lambda_{max}}$.
    \item Привести к виду, удобному для итераций.
    \begin{enumerate}
        \item $
        \left\{
        \begin{array}{lcl}
        x_1+3x_2+x_3=8\\
        2x_1+2x_2+x_3=9\\
        3x_1+x_2+2x_3=13\\
        \end{array}
        \right.
        $
        \item $
        \left\{
        \begin{array}{lcl}
        2x_1-4x_2+5x_3=3\\
        3x_1-3x_2-5x_3=-5\\
        4x_1+5x_2+2x_3=11\\
        \end{array}
        \right.
        $
    \end{enumerate}
\end{enumerate}
