
\newpage
\section{Лекция 16}
\subsection{Алгебраические зависимости в системах экономических показателей}
\textbf{Постановка задачи.}\\
Пусть $x_1,\cdots,x_n$ --- первичные показатели, а $y_1, \cdots, y_m$ --- расчетные (производные) показатели, $y_i=f_i(x_1,\cdots,x_n)$.\\
\\
Надо описать функциональные зависимости между показателями $y_1,\cdots,y_m$, то есть все такие функцие $\phi(z_1,\cdots,z_m)$, что $$\phi(y_1,\cdots,y_m)=0$$
Предположение:
$$y_i=f_i(x_1,\cdots,x_n)=\cfrac{p_i(x_1,\cdots,x_n)}{q_i(x_1,\cdots,x_n)} ~~-$$
дробно-рациональная функция, где $p_i, q_i$ --- многочлены от $n$ переменных.\\
Тогда можно считать функции $\phi$ многочленами от $m$ переменных.\\
Решение: для линейных многочленов $p_i, q_i$ --- Клейнер.\\
\\
\textbf{Пример 1.}\\
Для некоторого предприятия:\\
$x_1$ --- размер выручки предприятия от реализации продукции\\
$x_2$ --- издержки производства\\
$x_3$ --- размер капитала\\
$x_4$ --- численность занятых на предприятии\\
Расчетные показатели:\\
$y_1=\cfrac{x_1-x_2}{x_2}$ --- рентабельность производства\\
$y_2=\cfrac{x_1-x_2}{x_3}$ --- рентабельность капитала\\
$y_3=\cfrac{x_1}{x_3}$ --- средняя производительность капитала\\
$y_4=\cfrac{x_1}{x_4}$ --- средняя производительность труда\\
Решение по методу Клейнер: существует функциональная зависимость $$y_1y_2-y_1y_3+y_2=0$$
Любая другая полиномиальная зависимость имеет вид $$\phi(y_1, y_2, y_3)(y_1y_2-y_1y_3+y_2)=0$$
\\
\textbf{Пример 2.}\\
Сравнительный анализ производительности труда на двух предприятиях.\\
Первичные показатели:\\
$x_1$ --- доход первого предприятия\\
$x_2$ --- численность занятых на первом предприятии\\
$x_3$ --- доход второго предприятия\\
$x_4$ --- численность занятых на втором предприятии\\
Расчетные показатели:\\
$y_1=x_3\cfrac{x_2}{x_1}-x_4$ --- экономия затрат труда на втором предприятии по сравнению с первым\\
$y_2=\cfrac{x_3-x_1}{x_4-x_2}$ --- прирост (уменьшение) дохода, приходящийся на одногодополнительного занятого (высвобожденного) работника на втором предприятии по сравнению с первым\\
$y_3=x_3\cfrac{x_2}{x_4}-x_1$ --- часть прироста (уменьшения) дохода второго предприятия по сравнению с первым, обусловленная различием в их производительности труда\\
$y_4=\cfrac{x_3}{x_4}-\cfrac{x_1}{x_2}$ --- прирост производительности труда на втором предприятии по сравнению с первым\\
$y_5=\bigg(\cfrac{x_3}{x_4}/\Big(\cfrac{x_2}{x_1}-1\Big)\bigg)\cdot 100$ --- относительное (процентное) изменение производительности труда на втором предприятии по сравнению с первым\\
\\
\textbf{Пример 3.}\\
Анализ эффективности использования основных факторов производства.\\
Первичные показатели:\\
$x_1$ --- размер капитала предприятия\\
$x_2$ --- численность занятых на предприятии\\
Расчетные показатели:\\
$y_1$ --- рентабельность затрат на производство\\
$y_2$ --- рентабелность капитала\\
$y_3$ --- рентабельность труда\\
Выпуск и издержки предприятия описываются двумя производственными функциями от размеров труда и капитала: $$z=f(x_1, x_2),~u=g(x_1, x_2),$$
где $$f=a_0+a_1x_1+a_2x_2++a_{12}x_1x_2+a_{11}x_1^2+a_{22}x_2^2 ~~- $$
квадратичная функция с коэффициентами $a_0, a_1, \cdots, a_{22}$
$$g=b_0+b_1x_1+b_2x_2 ~~- $$
линейная функция с коэффициентами $b_0, b_1, b_2$.\\
Расчетные формулы:\\
$$y_1=\cfrac{f-g}{g}=\cfrac{a_0+a_1x_1+a_2x_2++a_{12}x_1x_2+a_{11}x_1^2+a_{22}x_2^2-b_1x_1-b_2x_2-b_0}{b_1x_1+b_2x_2+b_0}$$
$$y_2=\cfrac{f-g}{x_1}=\cfrac{a_0+a_1x_1+a_2x_2++a_{12}x_1x_2+a_{11}x_1^2+a_{22}x_2^2-b_1x_1-b_2x_2-b_0}{x_1}$$
$$y_3=\cfrac{f-g}{x_2}=\cfrac{a_0+a_1x_1+a_2x_2++a_{12}x_1x_2+a_{11}x_1^2+a_{22}x_2^2-b_1x_1-b_2x_2-b_0}{x_2}$$
\subsection{Описание алгоритма. Базис Гребнера}
Пусть $y_1=\cfrac{f_1}{g_1},~y_2=\cfrac{f_2}{g_2},\cdots,~y_m=\cfrac{f_m}{g_m}$, где $f_i, g_i$ --- многочлены от переменных $x_1, \cdots, x_n$, причем $$g=g_1\cdot g_2 \cdots g_m \neq 0$$
Введем новые многочлены от $n+m+1$ переменных $x_1, x_2, \cdots, x_n, z_1, z_2, \cdots, z_m, u$: $$h_1=g_1z_1-f_1$$
$$h_2=g_2z_2-f_2$$
$$\cdots$$
$$h_m=g_mz_m-f_m$$
$$h_{m+1}=g\cdot u-1$$
Отметим, что $h_j(x_1, x_2, \cdots, x_n, y_1, y_2, \cdots, y_m, g^{-1})=0$ для $j=1, \cdots, m+1$.\\
Пусть $I=\{\sum \limits_j c_j h_j\}$ (где $c_j$ --- многочлены от $x_1, x_2, \cdots, x_n, z_1, z_2, \cdots, z_m, u$) --- полиномиальный идеал, порожденный многочленами $h_j$.\\
\\
Будем сравнивать мономы от переменных $x_1, x_2, \cdots, x_n, z_1, z_2, \cdots, z_m, u$ лексикографически так, что $$u>x_n>x_{n-1}>\cdots >x_1>z_m>z_{m-1}>\cdots >z_1$$
Например, старший член многочлена $f=4x_2x_1+3z_1x_2+2u$ есть $\hat f=2u$.
\begin{definition}
    \textbf{Базисом Гребнера} идеала $I$ называется такое множество $G=\{q_1, q_2, \cdots\}$ элементов $I$, что старший член $\hat f$ любого элемента $f\in I$ делится на один из старших членов $\hat q_1, \hat q_2,\cdots$ элементов $G$.
    
    Базис Гребнера $G$ \textbf{редуцированный}, если ни один из $\hat q_i$ не делится на остальные.
\end{definition}
\begin{statement}
    Пусть $G=\{q_1, q_2, \cdots\}$ --- редуцированный базис Гребнера $I$, и пусть среди его элементов только $q_1, \cdots, q_k$ зависят от переменных $z_1, z_2, \cdots, z_m$. Тогда минимальный набор тождеств для $y_1, y_2, \cdots, y_m$:
$$q_1(y_1, y_2, \cdots, y_m)=0$$
$$\cdots$$
$$q_k(y_1, y_2, \cdots, y_m)=0$$
Если же таких множеств в $G$ нет, то показатели $y_1, y_2, \cdots, y_m$ --- независимы.\\
В этом случае любое другое полиномиальное соотношение имеет вид 
$$c_1q_1(y_1, y_2, \cdots, y_m)+\cdots+c_kq_k(y_1, y_2, \cdots, y_m),$$
где $c_i$ --- многочлены от переменных $y_1, y_2, \cdots, y_m$.
\end{statement}
\textbf{Пример 3 с частными значениями коэффициентов.}\\
Пусть $$f=2+x_1+x_2+3x_1x_2+4x_1^2+5x_2^2,$$
$$g=1+x_2+x_1$$
Тогда $$y_1=\cfrac{f-g}{g}=\cfrac{2+x_1+x_2+3x_1x_2+4x_1^2+5x_2^2-x_1-x_2-1}{x_1+x_1+1}$$
$$y_2=\cfrac{f-g}{x_1}=\cfrac{2+x_1+x_2+3x_1x_2+4x_1^2+5x_2^2-x_1-x_2-1}{x_1}$$
$$y_3=\cfrac{f-g}{x_2}=\cfrac{2+x_1+x_2+3x_1x_2+4x_1^2+5x_2^2-x_1-x_2-1}{x_2}$$
Последовательность расчетов: выписываем $h_1, \cdots, h_4$, потом строим редуцированный базис Гребнера идеала $I$ и получаем $G=\{q_1, \cdots, q_{17}\}$, где $$q_{17}=z_2^2z_3^2z_1-z_2z_3^2z_1^2-z_2^2z_3z_1^2+2z_2^2z_3z_1-5z_2z_3z_1^2-6z_2^2z_1^2+2z_2z_3^2z_1-z_2^2z_3^2-5z_3^2z_1^2$$
Можно выразить любой из $z_1, z_2, z_3$ через остальные...\\
\\
\textbf{Пример 3 в почти общем виде.}\\
Пусть $a_{11}=a_{22}=0$. Тогда
$$f=a_0+a_1x_1+a_2x_2+a_{12}x_1x_2$$
Тогда
$$y_1=\cfrac{f_1}{g_1}=\cfrac{a_0+a_1x_1+a_2x_2+a_{12}x_1x_2-b_1x_1-b_2x_2-b_0}{b_1x_1+b_2x_2+b_0}$$
$$y_2=\cfrac{f_2}{g_2}=\cfrac{a_0+a_1x_1+a_2x_2+a_{12}x_1x_2-b_1x_1-b_2x_2-b_0}{x_1}$$
$$y_3=\cfrac{f_3}{g_3}=\cfrac{a_0+a_1x_1+a_2x_2+a_{12}x_1x_2-b_1x_1-b_2x_2-b_0}{x_2}$$
и $$h_1=z_1g_1-f_1=z_1(b_1x_1+b_2x_2+b_0)-(a_0+a_1x_1+a_2x_2+a_{12}x_1x_2-b_1x_1-b_2x_2-b_0)$$
$$h_2=z_2g_2-f_2=z_2x_1-(a_0+a_1x_1+a_2x_2+a_{12}x_1x_2-b_1x_1-b_2x_2-b_0)$$
$$h_3=z_3g_3-f_3=z_3x_2-(a_0+a_1x_1+a_2x_2+a_{12}x_1x_2-b_1x_1-b_2x_2-b_0)$$
$$h_4=g\cdot u-1=u \cdot g_1\cdot g_2\cdot g_3-1=u(b_1x_1+b_2x_2+b_0)x_1x_2-1$$
Строим базис Гребнера с коэффициентами в поле $F=\mathbb{R}(a_0, a_1, a_2, a_{12}, b_0, b_1, b_2):$\\$G=\{q_1, \cdots, q_{16}\}$, причем
\begin{center}
    $q_{16}=(-b_2^2a_0+b_2b_0a_2)z_2^2z_1^2-b_2b_0z_3z_2^2z_1^2+(-a_0+b_0)z_3^2z_2^2+(-2b_2b_1a_0+b_1b_0a_2+b_2b_0a_1-$\\
    $-a_{12}b_0^2)z_3z_2z_1^2-b_0b_1z_3^2z_2z_1^2+(-b_2b_0+2b_2a_0-b_0a_2)z_3z_2^2z_1+(-b_0b_1+2b_1a_0-b_0a_1)z_3^2z_2z_1+$\\
    $+(-a_0b_1^2+b_0a_1b_1)z_3^2z_1^2+b_0z_3^2z_2^2z_1$
\end{center}
\textbf{Пример 2, решение (продолжение).}\\
$$y_1=\cfrac{x_3x_2-x_4x_1}{x_1}$$
$$y_2=\cfrac{x_3-x_1}{x_4-x_2}$$
$$y_3=\cfrac{x_3x_2-x_4x_1}{x_4}$$
$$y_4=\cfrac{x_3x_2-x_1x_4}{x_2x_4}$$
$$y_5=100~\cfrac{x_3x_2}{x_4x_1-1}$$
Упрощение:
$$y_1'=\cfrac{y_1}{y_3}=\cfrac{x_4}{x_1}$$
$$y_2'=y_2=\cfrac{x_3-x_1}{x_4-x_2}$$
$$y_3'=y_3=\cfrac{x_3x_2-x_4x_1}{x_4}$$
$$y_4'=\cfrac{y_3}{y_4}=x_2$$
$$y_5'=100~\cfrac{y_3}{y_4y_5}=\cfrac{x_1x_4-1}{x_3}$$
Тогда 
\begin{center}
    $q_{25}=-z_1+z_4^2+z_1^2z_4z_2^2z_5-z_1^2z_2z_3z_5+z_4z_5-z_1z_3z_5-z_1z_4^2z_2^2z_5^2+4z_1z_4^2z_2-2z_4z_1z_3+$\\
    $+4z_4z_5z_2z_1-2z_4^3z_2^2z_5z_1-2z_1^2z_2z_3z_4+z_1^2z_3z_4^2z_2^2z_5+z_1^2z_2^2z_5^2z_3z_4-z_3^2z_2z_4z_5z_1^2+z_3^2z_1^2+$\\
    $+z_3z_5^2z_2z_4z_1+z_4^2z_3z_5z_2z_1-z_1z_2^2z_4^2+z_1^2z_4^2z_2^2$
\end{center}
Для исходных переменных $y_1, \cdots, y_5$:
\begin{center}
    $0=-y_1y_4^4y_5^2+y_3^3y_4^2y_5^2+100y_1^2y_2^2y_3y_4^2y_5-100y_1^2y_2y_3y_4^3y_5+100y_4^3y_3^2y_5-100y_1y_3^2y_4^3y_5-$\\
    $-10000y_1y_4^4y_2^2+4y_1y_3^2y_2y_4^2y_5^2-2y_1y_3^2y_4^3y_5^2+400y_1y_3^2y_2y_4^2y_5-200y_1y_4^4y_2^2y_5-2y_1^2y_2y_3y_4^3y_5^2+$\\
    $+100y_1^2y_4^3y_2^2y_3y_5+10000y_1^2y_4^3y_2^2y_3-100y_4^3y_2y_1^2y_4^2y_5+y_1^2y_3y_4^4y_5^2+10000y_4^4y_2y_1y_3+$\\
    $+100y_4^3y_2y_1y_4y_5-y_1y_4^4y_2^2y_5^2+y_1^2y_2^2y_3y_4^2y_5^2$
\end{center}
\textbf{Пример 4.}\\
$$p_1=xy^2-y^3+1,~~p_2=x^2y+2xy-1$$
Надо построить базис Гребнера $G=\{p_1, p_2\}$.\\
Выберем лексикографический порядок, пусть $lex(y>x)$, то
$$y^3>xy^2>x^2y>xy>1$$
Пусть $p_1'=y^3-xy^2-1$.\\
$$p_1=LT(p_1)+\cdots,~~p_2=LT(p_2)+\cdots$$
$$L=lcm(LT(p_1), LT(p_2))$$
$$m_1=\cfrac{L}{LT(p_1)},~~m_2=\cfrac{L}{LT(p_2)}$$
($lcm$ --- НОК)\\
Тогда \textbf{S-полином} равен $S(p_1, p_2)=m_1p_1-m_2p_2$.\\
\\
У нас $$L=lcm(y^3, x^2y)=x^2y^3$$
$$S_1=S(p_1, p_2)=x^2(y^3-xy^2-1)-y^2(x^2y+2xy-1)=-x^3y^2-x^2-2xy^3+y^2$$
Разделим $S_1$ на $p_1'$, получим $$R(S_1)=S_1'=S_1-2xp_1'=x^3y^2-2x^2y^2+y^2-x^2-2x$$
Разделим $S_1'$ на $p_2$, получим
$$R(S_1')=S_1''=S_1'-xyp_2=y^2+xy-x^2-2x$$
И так далее.
\subsection{Домашнее задание 16}\begin{enumerate}
    \item Решить с помощью базисов Грёбнера систему уравнений\\ \\
    $
    \left\{
    \begin{array}{lcl}
    x^2y+x^2z-2xz=0\\
    x^2+2yz-3=0\\
    x^4-y^2z^2=0\\
    \end{array}
    \right.
    $
    \item Найти базис Грёбнера\\ \\
    $
    \left\{
    \begin{array}{lcl}
    f_1=x^3y+2xy^3-3x^2y^2\\
    f_2=xy^2-2y^2+x-1\\
    f_3=x^4\\
    \end{array}
    \right.
    $
    \\ \\и решить систему уравнений $f_1=f_2=f_3=0$.
\end{enumerate}