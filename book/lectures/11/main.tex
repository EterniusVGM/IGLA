\newpage
\section{Лекция 11}
\subsection{Решение систем линейных уравнений}
Пусть дана система уравнений:\begin{center}
    $
    \left\{
    \begin{array}{lcl}
    x+0.99y=1.01 \\
    x+1.01y=0.99 \\
    \end{array}
    \right.
    $
\end{center}
Надо найти приблизительное решение. Все коэффициенты известны с точностью до 1\%.
Либо $x+y=1$ --- бесконечное множество решений, либо 
$
\left\{
\begin{array}{lcl}
x+y=1.01 \\
x+y=0.99 \\
\end{array}
\right.
$
--- нет решений.\\
$$A\bar x=\bar b$$
Если считать, что коэффициенты точно известны, то можно найти решение.
\[\begin{pmatrix}[r]
1 & 0.99 \\
1 & 1.01 \\
\end{pmatrix} \begin{pmatrix}[r]
x \\
y \\
\end{pmatrix} = \begin{pmatrix}[r]
1.01 \\
0.99 \\
\end{pmatrix}\]
\[\bar x= A^{-1} \bar b = \cfrac{1}{0.02} \begin{pmatrix}[r]
1.01 & -0.99 \\
-1 & 1 \\
\end{pmatrix} \begin{pmatrix}[r]
1.01 \\
0.99 \\
\end{pmatrix} = \begin{pmatrix}[r]
2 \\
-1 \\
\end{pmatrix}\]
При малом изменении коэффициентов (даже на 1\%) решение может испортиться.\\ \\
$
\left\{
\begin{array}{lcl}
x+y=2 \\
x+2y=3 \\
\end{array}
\right.
$
--- устойчивая система.\\ \\
Коэффициенты известны с точностью до 1\%.\\
\[\begin{pmatrix}[r]
x \\
y \\
\end{pmatrix} = \begin{pmatrix}[r]
1 \\
1 \\
\end{pmatrix} \pm 2\% \]
Даже если $b$ известен точно, при изменении матрицы все может измениться. Есть два типа ошибок --- неточная матрица и неточная правая часть.\\

    \textbf{Общая постановка задачи.} 
    
    Найти $\bar x$, удовлетворяющий системе $A\bar x=\bar b$.\\
    Пусть $\hat A \hat x=\hat b$ --- решение приближенной системы ($\hat A, \hat x, \hat b$ считаем известны).\\
    Обозначим: $$\Delta A =\hat A-A$$
    $$\Delta x = \hat x-x$$
    $$\Delta b=\hat b -b$$
    Мы хотим оценить $\Delta x$, причем $\Delta b$ считаем "малыми".\\
    Так как $A$ и $b$ неизвестны, считаем $x \approx \hat x$.\\
    Абсолютная погрешность: $|\Delta x|$\\
    Относительная погрешность: $\cfrac{|\Delta x|}{|x|}$, где $|\cdot|$ --- неоторая векторная норма.\\
    Оценивая $|b|_1 (|b|_{\infty})$ можем оценить $|\Delta x|_1 (|\Delta x|_{\infty})$.\\


\textbf{Упрощенный вариант $\hat A=A$}\\
Дано:\begin{center}
    $
    \left\{
    \begin{array}{lcl}
    Ax=b \\
    A(x+\Delta x)=b+\Delta b \\
    \end{array}
    \right.
    $ 
    $\Leftrightarrow$
    $
    \left\{
    \begin{array}{lcl}
    Ax=b & (*) \\
    A\Delta x=\Delta b & (**)\\
    \end{array}
    \right.
    $
    $\Leftrightarrow$
    $
    \left\{
    \begin{array}{lcl}
    x=A^{-1}b & $(v)$ \\
    \Delta x=A^{-1} \Delta b & $(vv)$\\
    \end{array}
    \right.
    $
\end{center}
Считаем, что $|x|\approx |\hat x|,~ |b|\approx |\hat b|$.\\
Из (*) получим $$|b|\leqslant ||A||~|x| \Leftrightarrow |x| \geqslant \cfrac{|b|}{||A||}~~(1),$$а из (**) $$|\Delta b|\leqslant ||A||~|\Delta x|~~(2)$$
Из (v) получим $$|x|\leqslant ||A^{-1}||~|b|~~(3),$$ а из (vv) $$|\Delta x|\leqslant ||A^{-1}||~|\Delta b|~~(4)$$
Тогда относительная погрешность: $$\delta x=\cfrac{|\Delta x|}{|x|}\overset{(1), (4)}{\leqslant} \cfrac{||A^{-1}||~|\Delta b|}{|b|/||A||} = ||A||~||A^{-1}||\cfrac{|\Delta b|}{|b|}$$
$$\delta x \overset{(2), (3)}{\geqslant} \cfrac{|\Delta b|/||A||}{||A^{-1}||~|b|}=\cfrac{1}{||A||~||A^{-1}||}~\cfrac{|\Delta b|}{|b|}$$
\begin{definition}
    $||A||~||A^{-1}|| = cond(A) = \chi (A) $ --- \textbf{число обусловленности}.
\end{definition}
Например, для евклидовой нормы $$\chi_2(A)=||A||_2~||A^{-1}||_2$$
В итоге, получим: $$\cfrac{1}{\chi(A)}\delta b \leqslant \delta x\leqslant \chi(A) \delta b,~~\delta b=\cfrac{|\Delta b|}{|b|}$$
А для общей задачи, если $(A+\Delta A) \hat x =b+\Delta b,~ \Delta x=\hat x-x$ ($\Delta b, \Delta A, \Delta x$ --- малые), то
$$\cfrac{1}{\chi(A)}(\delta b+\delta A) \leqslant \delta x \leqslant \chi(A)(\delta b+\delta A),~~\delta A=\cfrac{||\Delta A||}{||A||}$$
\textbf{Пример 1.}\\
\[A=\begin{pmatrix}[r]
1 & 0.99 \\
1 & 1.01 \\
\end{pmatrix},~ \bar b=\begin{pmatrix}[r]
1.01 \\
0.99 \\
\end{pmatrix}\]
Пусть у нас норма $|\cdot|_1$, тогда число обусловленности 
\[\chi_1(A)=||A||_1||A^{-1}||_1=\begin{Vmatrix}[r]
1 & 0.99 \\
1 & 1.01 \\
\end{Vmatrix}_1 \cdot 50 \cdot \begin{Vmatrix}[r]
1.01 & -0.99 \\
-1 & 1 \\
\end{Vmatrix}_1 = 2\cdot 50 \cdot 2.01 = 201\]
$$\delta x \leqslant 201 \delta b$$
Чтобы было $\delta x <1\%$ надо $\delta b<\cfrac{1}{201}\cdot 1\%$.\\
\\
\textbf{Пример 2.}\\
\[A=\begin{pmatrix}[r]
1 & 1 \\
1 & 2 \\
\end{pmatrix},~ \bar b=\begin{pmatrix}[r]
2 \\
3 \\
\end{pmatrix}, ~\bar x=\begin{pmatrix}[r]
1 \\
1 \\
\end{pmatrix}\]
У $b$ ошибка в 1\%. На сколько изменится $x$ при изменении $b$?\\ \\
Найдем число обусловленности.
\[\chi_{\infty}(A) = \begin{Vmatrix}[r]
1 & 1 \\
1 & 2 \\
\end{Vmatrix}_{\infty} \begin{Vmatrix}[r]
2 & -1 \\
-1 & 1 \\
\end{Vmatrix}_{\infty} = 3\cdot 3 = 9\]
$\delta x < 1\%$, если $\delta b < 0.1 \%$.
\begin{lemma}[Свойства числа обусловленности]
    \ 
\begin{enumerate}
    \item $\chi(A) \geqslant 1$ 
    \item $\chi(AB)\leqslant \chi(A) \chi(B)$
    \item $\chi(A^{-1})=\chi(A)$
    \item Для евклидовой нормы $||\cdot||_2$: если $\sigma_1\geqslant \cdots \geqslant \sigma_n$ --- сингулярные собственные значения, что $$\sigma_1=\sqrt{\lambda_{max}(A^*A)},$$
    $$\sigma_n=\sqrt{\lambda_{min}(A^*A)},$$
    тогда $\chi_2(A)=||A||_2||A^{-1}||_2=\sigma_1(A)\sigma_1(A^{-1})=\cfrac{\sigma_1}{\sigma_n}$\\
    Например, если $A=A^*$ --- самосопряженная матрица, то $$\chi_2(A)= \bigg | \cfrac{ \lambda_{max}(A)}{ \lambda_{min}(A)} \bigg |$$
    \item Для любой матричной нормы $||\cdot||$, согласованной с некоторой векторной нормой $|\cdot|$ $$\chi(A)\geqslant \bigg | \cfrac{ \lambda_{max}(A)}{ \lambda_{min}(A)} \bigg |$$
\end{enumerate}
\end{lemma}
\begin{proof}
    \ 
    \begin{enumerate}
        \item Неравенство выполнено, так как для любой нормы $||A^{-1}||\geqslant \cfrac{1}{||A||}$. Кроме того, существует такая матрица $A$, что $\chi(A)=1$ только, если $||E||=1$ --- норма сохраняет единицу, так как иначе $\chi(A)=||A||~||A^{-1}||\geqslant ||AA^{-1}||=||E||$
        \item $\chi(A) \chi(B)=||A||~||A^{-1}||~||B||~||B^{-1}|| \geqslant ||AB||~||B^{-1}A^{-1}||=||AB||~||(AB)^{-1}||=\chi(AB)$
        \item Очевидно.
        \item Проверка по определению.
        \item $||A||\geqslant \rho(A)=|\lambda_{max}(A)|$\\
        $||A^{-1}||\geqslant \rho(A^{-1})=\bigg | \cfrac{1}{\lambda_{min}(A)} \bigg | \Rightarrow \chi(A)=||A||~||A^{-1}||\geqslant \cfrac{|\lambda_{max}(A)|}{|\lambda_{min}(A)|}$\\
        Так как $\lambda$ --- собственное значение $A$, то $\cfrac{1}{\lambda}$ --- собстенное значение $A^{-1}.$
    \end{enumerate}
\end{proof}
\textbf{Пример 3.}\\
\[\begin{pmatrix}[r]
0.97 & 1.98 \\
0.99 & 3.02 \\
\end{pmatrix} \begin{pmatrix}[r]
x_1 \\
x_2 \\
\end{pmatrix} = \begin{pmatrix}[r]
3.01 \\
3.97 \\
\end{pmatrix}\]
Решить приближенно и оценить погрешность решения.\\
\[\hat A=\begin{pmatrix}[r]
1 & 2 \\
1 & 3 \\
\end{pmatrix},~\hat b= \begin{pmatrix}[r]
3 \\
4 \\
\end{pmatrix} \Rightarrow \hat x = \begin{pmatrix}[r]
1 \\
1 \\
\end{pmatrix}\]
Относительно какой нормы ошибка меньше? ($|\cdot|_1, |\cdot|_{\infty}, |\cdot|_2$)\\
\[\chi(A)\approx \chi(\hat A)\approx ||\hat A||~||\hat A^{-1}||=\begin{Vmatrix}[r]
1 & 2 \\
1 & 3 \\
\end{Vmatrix} \begin{pmatrix}[r]
3 & -2 \\
1 & 1 \\
\end{pmatrix}\]
\begin{center}
    $\chi_{\infty}(A)=max\{3, 4\}\cdot max\{5, 2\}=20$\\
    $\chi_1(A)=max\{2, 5\}\cdot max\{4, 3\}=20$\\
    $\chi_2(A)=\sqrt{\cfrac{\lambda_{max}(A^*A)}{\lambda_{min}(A^*A)}}$
\end{center}
\[A^*A=\begin{pmatrix}[r]
1 & 1 \\
2 & 3 \\
\end{pmatrix} \begin{pmatrix}[r]
1 & 2 \\
1 & 3 \\
\end{pmatrix} = \begin{pmatrix}[r]
2 & 5 \\
5 & 13 \\
\end{pmatrix}\]
Собственные значения $A^*A$: $\lambda_{max}=14.93,~ \lambda_{min}=0.069$, тогда $\chi_2(A)=\sqrt{\cfrac{14.93}{0.069}}\approx 15$
\[\Delta b=\hat b-b =\begin{pmatrix}[r]
-0.01 \\
0.03 \\
\end{pmatrix}\]
\begin{center}
    $\delta_1 b=\cfrac{|\Delta b|_1}{|b|_1}\approx \cfrac{|\Delta b|_1}{|\hat b|_1}=\cfrac{0.01+0.03}{3+4}=0.0057$\\
    $\delta_2 b=\cfrac{\sqrt{0.01^2+0.03^2}}{\sqrt{3^2+4^2}}=0.0063$\\
    $\delta_{\infty}b=\cfrac{0.03}{4}=0.0075$\\
    $\delta_1 A=\cfrac{||\Delta A||_1}{||A||_1}=\cfrac{0.04}{5}=0.008$\\
    $\delta_{\infty} A =\cfrac{||\Delta A||_{\infty}}{||A||_{\infty}}=\cfrac{0.05}{4}=0.0125$
\end{center}
Собственные значения для $\Delta A^* \Delta A$: $\lambda_{min}=0.000487,~ \lambda_{max}=0.00131, \chi_2(\Delta A)=\\=\sqrt{\cfrac{0.00131}{0.000487}}=1.64$, тогда
\begin{center}
    $\delta_2 A =\cfrac{||\Delta A||_2}{||A||_2}=\cfrac{\sigma_1(\Delta A)}{\sigma_1(A)}=\cfrac{\sqrt{0.00131}}{\sqrt{14.93}}=0.009$\\
    $\delta_1 x\leqslant \chi_1(A)(\delta_1 b+\delta_1 A) =20\cdot (0.0057+0.008)=0.27$\\
    $\delta_2 x\leqslant \chi_2(A)(\delta_2 b+\delta_2 A) =15\cdot (0.0063+0.009)=0.23$\\
    $\delta_{\infty} x\leqslant \chi_{\infty}(A)(\delta_{\infty} b+\delta_{\infty} A) =20\cdot (0.0075+0.0125)=0.4$
\end{center}
Норма $|\cdot|_2$ дала наименьшее значение ошибки 0.23, а $|\cdot|_{\infty}$ --- наибольшее 0.4.
\subsection{Ошибка для обратной матрицы}
$A=\hat A+\varepsilon$, если знаем $\hat A^{-1}$. Ошибка приближения $A^{-1}\approx \hat A^{-1}$?\\
$$\chi(A)\approx \chi(\hat A)$$
Надо оценить
$$\delta A^{-1}=\cfrac{||(\hat A+\varepsilon)^{-1}-A^{-1}||}{||A^{-1}||}=\cfrac{||-\hat A^{-1}(E-\hat A(\hat A+\varepsilon)^{-1})||}{||A^{-1}||}=\cfrac{||-\hat A^{-1}(E-(E-\hat A^{-1}\varepsilon)^{-1})||}{||A^{-1}||}\approx$$ $$\approx \cfrac{||\hat A^{-1}(E-(E+\hat A^{-1}\varepsilon)^{-1})||}{||\hat A^{-1}||}\leqslant \cfrac{||\hat A^{-1}||}{||\hat A^{-1}||}~||E-(E+\hat A^{-1}\varepsilon)^{-1}||$$
$$E-(E-\hat A^{-1} \varepsilon)^{-1}=E-(E-Y)^{-1}=E-(E+Y+Y^2+Y^3+\cdots)=-(Y+Y^2+\cdots)$$
$$||\sum\limits_{t=1}^{\infty}(\hat A^{-1}\varepsilon)^t||\leqslant\sum\limits_{t=1}^{\infty}||\hat A^{-1}||^t||\varepsilon||^t=\cfrac{||\hat A^{-1}||~||\varepsilon||}{1-||\hat A^{-1}||~||\varepsilon||}\approx||\hat A^{-1}||~||\varepsilon||=\cfrac{\chi(\hat A)}{||\hat A||}||\varepsilon||=\chi(\hat A)\cfrac{||\varepsilon||}{||\hat A||}$$
Более точно $$\delta A^{-1}\leqslant \cfrac{\chi(\hat A)\delta \varepsilon}{1-\chi(\hat A)\delta \varepsilon},~~\delta \varepsilon=\cfrac{||\varepsilon||}{||\hat A||}$$
\textbf{Пример 4.}\\
\[A=\begin{pmatrix}[r]
2 & 0 & 0 \\
0 & 1 & 1 \\
0 & 0 & 1 \\
\end{pmatrix}\]
Найти $f(A)=A^{100},~ sin\bigg(\cfrac{\pi}{2}~A\bigg)$.\\
Минимальный многочлен для матрицы $\varphi(\lambda)=(\lambda-1)^2(\lambda-2)$. Тогда для любой функции $f(\lambda)$, определенной на спектре матрицы $A$ спектральное разложение будет иметь вид $$f(A)=f(1)z_{11}+f'(1)z_{12}+f(2)z_{21}$$
\begin{center}
    $f_1=\lambda^2-2\lambda+1$\\
    $f_1(A)=f_1(2)z_{21}$\\
    $A^2-2A+E=(2^2-2\cdot 2+1)z_{21}$
\end{center}
Получим, что \[z_{21}\begin{pmatrix}[r]
1 & 0 & 0 \\
0 & 0 & 0 \\
0 & 0 & 0 \\
\end{pmatrix}\]
\begin{center}
    $f_2=\lambda-2$\\
    $A-2E=(-1)z_{11}+1\cdot z_{12}$\\
    $f_3=1$\\
    $E=1\cdot z_{21}+1\cdot z_{11}$
\end{center}
Получим, что \[z_{11}\begin{pmatrix}[r]
0 & 0 & 0 \\
0 & 1 & 0 \\
0 & 0 & 1 \\
\end{pmatrix}\]
\[z_{12}\begin{pmatrix}[r]
2 & 0 & 0 \\
0 & 1 & 1 \\
0 & 0 & 1 \\
\end{pmatrix} - \begin{pmatrix}[r]
2 & 0 & 0 \\
0 & 2 & 0 \\
0 & 0 & 2 \\
\end{pmatrix} + \begin{pmatrix}[r]
0 & 0 & 0 \\
0 & 1 & 0 \\
0 & 0 & 1 \\
\end{pmatrix} = \begin{pmatrix}[r]
0 & 0 & 0 \\
0 & 0 & 1 \\
0 & 0 & 0 \\
\end{pmatrix}\]
Для $A^{100}$: $f(1)=1^{100},~f'(1)=100\cdot 1^{99},~f(2)=2^{100}$\\
Получим
\[A_{100}=1^{100}\begin{pmatrix}[r]
0 & 0 & 0 \\
0 & 1 & 0 \\
0 & 0 & 1 \\
\end{pmatrix} + 100\cdot 1^{99}\begin{pmatrix}[r]
0 & 0 & 0 \\
0 & 0 & 1 \\
0 & 0 & 0 \\
\end{pmatrix} + 2^{100}\begin{pmatrix}[r]
1 & 0 & 0 \\
0 & 0 & 0 \\
0 & 0 & 0 \\
\end{pmatrix} = \begin{pmatrix}[r]
2^{100} & 0 & 0 \\
0 & 1 & 100 \\
0 & 0 & 1 \\
\end{pmatrix}\]
Аналогично получим
\[sin\bigg(\cfrac{\pi}{2}~A\bigg)=\begin{pmatrix}[r]
sin~\pi & 0 & 0 \\
0 & sin~\cfrac{\pi}{2} & \cfrac{\pi}{2}~cos~\cfrac{\pi}{2} \\
0 & 0 & sin~\cfrac{\pi}{2} \\
\end{pmatrix} = \begin{pmatrix}[r]
0 & 0 & 0 \\
0 & 1 & 0 \\
0 & 0 & 1 \\
\end{pmatrix}\]
\subsection{Домашнее задание  11}\begin{enumerate}
    \item Доказать утверждение из лекции. Если $(A+\Delta A)\hat x=b+\Delta b,~\Delta x=\hat x-x$, то 
    $$\cfrac{1}{\chi(A)}(\delta b+\delta A) \leqslant \delta x \leqslant \chi(A)(\delta b+ \delta A),$$ где $\delta A=\cfrac{||\Delta A||}{||A||}$ для малых $\Delta b, \Delta A, \Delta x$.
    \item Вычислить $ln A$, где 
    \[A = \begin{pmatrix}[r]
    0 & 1 & 0\\
    -4 & 4 & 0\\
    -2 & 1 & 2\\
    \end{pmatrix}\]
    \item Дана система уравнений\\ \\
    $
    \left\{
    \begin{array}{lcl}
    x_1+x_2=3\\
    x_1-x_2=4\\
    \end{array}
    \right.
    $
    \\ \\
    Элементы главной диагонали могут меняться на $\varepsilon_1$, а элементы правой части на $\varepsilon_2$. Оценить возможное изменение решения для нормы $||A||=max\sqrt{\lambda^*\lambda}$.
    \item Решить пример Крылова.($\sqrt{7}$ берется с разной точностью).\\ \\
    $
    \left\{
    \begin{array}{lcl}
    x_1+2x_2=3\\
    \sqrt{7}x_1+2\sqrt{7}x_2=3\sqrt{7}\\
    \end{array}
    \right.
    $
    \\ 
\end{enumerate}
